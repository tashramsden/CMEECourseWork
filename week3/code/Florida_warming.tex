\documentclass[12pt]{article}

\usepackage[a4paper, total={6in, 10.5in}]{geometry}

\usepackage{graphicx}
\graphicspath{{../results/}}

\usepackage{xcolor}
\usepackage{fancyhdr}
\pagestyle{fancy}
\fancyhead[L,R,C]{}
\fancyfoot[L,C]{}
\fancyfoot[R]{\color{lightgray}Tash Ramsden}
\renewcommand{\headrulewidth}{0pt}
\renewcommand{\footrulewidth}{0pt}

\usepackage{caption}
\captionsetup{font=footnotesize}

\title{\vspace{-1.5em}{Is Florida getting warmer?\vspace{-2em}}}
\date{}
\author{}

\begin{document}
    \maketitle

    \thispagestyle{fancy}

    Pearson's product-moment correlation was used to analyse the relationship between temperature and time from the year 1901 to 2000 in Florida ($\rho$ = 0.53) (Fig. \ref{fig:florida_data}).
    
    \begin{figure}[h!]
        \centering
        \includegraphics[width=0.7\linewidth]{florida_data.pdf}
        \caption{Temperature ($^{\circ}$C) over time in Florida.}
        \label{fig:florida_data}
    \end{figure}

    A permutation test with 10,000 replicates was carried out in which temperature was randomised and the correlation coefficients calculated. A permutation test was implemented in order to account for the non-independence of temperature points over time.

    \begin{figure}[h!]
        \centering
        \includegraphics[width=0.7\linewidth]{florida_coefs_hist.pdf}
        \caption{Distribution of correlation coefficients for temperature explained by time from 10,000 randomised samples of temperature. The dashed red line shows the observed correlation coefficient from the original Florida temperature dataset ($\rho$ = 0.53, p $<$ 0.01).}
        \label{fig:coefs_hist}
    \end{figure}

    Figure \ref{fig:coefs_hist} shows the distribution of the values of these randomised correlation coefficients, with the observed correlation coefficient highlighted in red. The observed correlation coefficient is significantly higher than the randomised values (p $<$ 0.01). This demonstrates that there is a significant positive correlation between temperature in Florida and time, even when non-independence is accounted for; Florida is getting warmer.

\end{document}
